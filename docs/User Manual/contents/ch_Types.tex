\chapter{Types}

This chapter describes all the data types defined by and used throughout the OS. All type-definitions are suffixed with \_t. 

\section{List of types}

\begin{pdtable}{tbl:ostypes}{Available types}
\begin{pdtabular}{l|p{10cm}}
	\pdtablehead{\textbf{Definition} & \textbf{Description}}
	\pdtablerow{\textit{U8\_t} & Unsigned 8-bit integer.}
	\pdtablerow{\textit{U16\_t} & Unsigned 16-bit integer.}
	\pdtablerow{\textit{U32\_t} & Unsigned 32-bit integer.}
	\pdtablerow{\textit{S8\_t} & Signed 8-bit integer.}
	\pdtablerow{\textit{S16\_t} & Signed 16-bit integer.}
	\pdtablerow{\textit{S32\_t} & Signed 32-bit integer.}
	\pdtablerow{\textit{MemBase\_t} & RAM Memory base type. Width is defined by \TODO{ref} \textit{CONFIG\_MEM\_WIDTH\_X\_BITS}.}
	\pdtablerow{\textit{Id\_t} & Object identifier. Used to reference any OS object, more information can be found in \cref{tbl:idtype}.}
	\pdtablerow{\textit{IdType\_t} & ID type. Indicates what type of object the ID is belongs to.}
	\pdtablerow{\textit{Prio\_t} & Priority type. Valid range: 0-24.}
	\pdtablerow{\textit{OsVer\_t} & OS Version. Format: 0x0041 = V 0.4.1}
	\pdtablerow{\textit{OsResult\_t} & System call result. More info see \TODO{ref}.}	
	\pdtablerow{\textit{Task\_t} & Task handler entry point. Each task is assigned a handler function
		of type \textit{Task\_t} upon creation. This handler will be called by the kernel. }

\end{pdtabular}
\end{pdtable}

\section{The Object ID}

All objects existing contained the OS lists are assigned an identifier or ID for short. This object ID is passed when making a system call to indirectly reference the meant object.\\
The ID type is a 16-bits unsigned integer, in most cases, with the exception of Event IDs, consisting of the two parts shown in \cref{tbl:idtype}. 

\begin{pdtable}{tbl:idtype}{ID type parts}
	\begin{pdtabular}{l|c|p{9cm}}
		\pdtablehead{\textbf{Name} & \textbf{Bit range} &\textbf{Remarks}}
		\pdtablerow{Object Type & 15:12 & Indicates the object type which the ID is assigned to. See \cref{tbl:objtype}.}
		\pdtablerow{Unique ID & 11:0 & Unique identifier within this object type.}		
	\end{pdtabular}
\end{pdtable}

\begin{pdtable}{tbl:objtype}{Object type}
	\begin{pdtabular}{l|c}
		\pdtablehead{\textbf{Name} & \textbf{Value}}
		\pdtablerow{Memory pool & 0x0 }
		\pdtablerow{Task & 0x1 }		
		\pdtablerow{Timer & 0x2 }
		\pdtablerow{Eventgroup & 0x3 }
		\pdtablerow{Semaphore & 0x4 }
		\pdtablerow{Mailbox & 0x5 }
		\pdtablerow{Ringbuffer & 0x6 }
						
	\end{pdtabular}
\end{pdtable}

\section{The OS result type}
	
	
	