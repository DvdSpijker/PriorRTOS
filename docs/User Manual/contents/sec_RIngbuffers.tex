\section{\textit{Ringbuffer} API}

\subsection{RingbufCreate}
\label{func:RingbufCreate}
\begin{pdfunction}
{RingbufCreate(U32\_t size) }
{ 
Creates a ring-buffer of given size. Note that the width of 
element in the buffer is defined by RingbufBase\_t (U8\_t by default). }
\pdfunctionargin{(U32_t) size}{ Ring-buffer size in elements. }
\pdfunctionreturn{(Id_t) }{Ring-buffer ID }{INVALID_ID}{if an error occurred during creation. }
\pdfunctionreturn{(Id_t) }{Ring-buffer ID }{Other}{valid ID if the ring-buffer was created. }
\end{pdfunction}

\subsection{}
\label{func:}
\begin{pdfunction}
{RingbufDelete (Id\_t ringbuf\_id) }
{ 
Deletes the specified ring-buffer. ringbuf\_id is set to INVALID\_ID if 
operation is successful. }
\pdfunctionargin{(Id_t *) ringbuf_id}{ ID of the ring-buffer to delete. }
\pdfunctionreturn{(OsResult_t) }{sys call result }{OS_OK}{if the ring-buffer was deleted. }
\pdfunctionreturn{(OsResult_t) }{sys call result }{OS_ERROR}{if the ring-buffer could not be found. }
\end{pdfunction}

\subsection{RingbufWrite}
\label{func:RingbufWrite}
\begin{pdfunction}
{RingbufWrite(Id\_t ringbuf\_id, RingbufBase\_t data,  U32\_t length) }
{ 
Writes a given number of data elements to the ring-buffer. The 
of elements actually written is returned. }
\pdfunctionargin{(Id_t) ringbuf_id}{ Ring-buffer ID. }
\pdfunctionargin{(RingbufBase_t) data}{ Array of data elements. }
\pdfunctionargin{(U32_t *) length}{ Length of data in elements. }
\pdfunctionreturn{(OsResult_t) }{sys call result }{OS_OK}{if data was written. }
\pdfunctionreturn{(OsResult_t) }{sys call result }{OS_FAIL}{if the buffer is empty. }
\pdfunctionreturn{(OsResult_t) }{sys call result }{OS_LOCKED}{if the ringbuffer is already locked for writing. }
\pdfunctionreturn{(OsResult_t) }{sys call result }{OS_ERROR}{if the ringbuffer could not be found. }
\end{pdfunction}

\subsection{RingbufRead}
\label{func:RingbufRead}
\begin{pdfunction}
{RingbufRead(Id\_t ringbuf\_id, RingbufBase\_t data,  U32\_t amount) }
{ 
Reads a given number of data elements from the ring-buffer. The 
of elements actually read is returned. Read data is copied to the 
target array. The target array has to comply with the pre-conditions 
below. }
\pdfunctionargin{(Id_t) ringbuf_id}{ Ring-buffer ID. }
\pdfunctionargin{(RingbufBase_t) target}{ Target data array. 
                         Pre-condition: target[0] = 0xFE, target[amount-1] = 0xEF}
\pdfunctionargin{(U32_t *) amount}{ Amount of data to read in elements. }
\pdfunctionreturn{(OsResult_t) }{sys call result }{OS_OK}{if data was read. }
\pdfunctionreturn{(OsResult_t) }{sys call result }{OS_LOCKED}{if the ringbuffer is already locked for reading. }
\pdfunctionreturn{(OsResult_t) }{sys call result }{OS_FAIL}{if the buffer is empty. }
\pdfunctionreturn{(OsResult_t) }{sys call result }{OS_OUT_OF_BOUNDS}{if the target array was not compliant. }
\pdfunctionreturn{(OsResult_t) }{sys call result }{OS_ERROR}{if the ringbuffer could not be found. }
\end{pdfunction}

\subsection{RingbufDump}
\label{func:RingbufDump}
\begin{pdfunction}
{RingbufDump(Id\_t ringbuf\_id, RingbufBase\_t target) }
{ 
Dumps all data present in the ring-buffer in the target array. 
number of elements actually read is returned. The target array has to comply 
the pre-conditions stated below. }
\pdfunctionargin{(Id_t) ringbuf_id}{ Ring-buffer ID. }
\pdfunctionargin{(RingbufBase_t) target}{ Target data array. 
                         Pre-condition: target[0] = 0xFE,
                         target[ring-buffer size-1] = 0xEF}
\pdfunctionreturn{(U32_t) }{Actual amount of elements read from the ring-buffer. 0 if }{}{}
\end{pdfunction}

\subsection{RingbufFlush}
\label{func:RingbufFlush}
\begin{pdfunction}
{RingbufFlush(Id\_t ringbuf\_id) }
{ 
Resets the ring-buffer's read and write locations as well as 
current data count resulting in all its initial space becoming available. }
\pdfunctionargin{(Id_t) ringbuf_id}{ Ring-buffer ID. }
\pdfunctionreturn{(OsResult_t) }{sys call result }{OS_OK}{if the ring-buffer was flushed. }
\pdfunctionreturn{(OsResult_t) }{sys call result }{OS_LOCKED}{if the ring-buffer is locked for reading or writing. }
\pdfunctionreturn{(OsResult_t) }{sys call result }{OS_ERROR}{if the ring-buffer could not be found. }
\end{pdfunction}

\subsection{RingbufSearch}
\label{func:RingbufSearch}
\begin{pdfunction}
{RingbufSearch(Id\_t ringbuf\_id, RingbufBase\_t query,  U32\_t query\_length) }
{ 
Searched the ring-buffer for the given query of given length. 
number of occurrences of the query in the buffer is returned. }
\pdfunctionargin{(Id_t) ringbuf_id}{ Ring-buffer ID. }
\pdfunctionargin{(RingbufBase_t) query}{ Search query. }
\pdfunctionargin{(U32_t) query_length}{ Length of the search query. }
\pdfunctionreturn{(U32_t) }{Number of occurrences. }{}{}
\end{pdfunction}

\subsection{RingbufDataCountGet}
\label{func:RingbufDataCountGet}
\begin{pdfunction}
{RingbufDataCountGet(Id\_t ringbuf\_id) }
{ 
Returns the amount of data elements are present in the ring- 
buffer.}
\pdfunctionargin{(Id_t) ringbuf_id}{ Ring-buffer ID. }
\pdfunctionreturn{(U32_t) }{Number of data elements. }{}{}
\end{pdfunction}

\subsection{RingbufDataSpaceGet}
\label{func:RingbufDataSpaceGet}
\begin{pdfunction}
{RingbufDataSpaceGet(Id\_t ringbuf\_id) }
{ 
Returns the amount of space left in the ring-buffer. }
\pdfunctionargin{(Id_t) ringbuf_id}{ Ring-buffer ID. }
\pdfunctionreturn{(U32_t) }{Data space left. }{}{}
\end{pdfunction}
