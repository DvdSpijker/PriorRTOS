\chapter{Events}

\section{Publishing}
Events are used throughout the OS to communicate the creation and deletion of objects, object access, state changes, exceptions. All event types can be viewed in \cref{tbl:eventtypes}.\\
Typically events are generated or Published by OS objects such as Timers (\TODO{ref}). 


\begin{pdtable}{tbl:eventtypes}{Event types}
	\begin{pdtabular}{l|c|p{9.5cm}}
		\pdtablehead{\textbf{Macro}  & \textbf{Event type} & \textbf{Description}}
		\pdtablerow{\textit{EVENT\_TYPE\_ACCESS} & Access & Events related to read/write access of a particular object.}
		\pdtablerow{\textit{EVENT\_TYPE\_STATE\_CHANGE} & State change & Events related to state changes of a particular object. E.g. Timer transitioned from Running state to Stopped state.}
		\pdtablerow{\textit{EVENT\_TYPE\_CREATE} & Create & Published when an object of the specified type was created.}
		\pdtablerow{\textit{EVENT\_TYPE\_DELETE}  & Delete & Published when a particular object was deleted. }
		\pdtablerow{\textit{EVENT\_TYPE\_EXCEPTION} & Exception & Published when an exception is thrown.}
	\end{pdtabular}
\end{pdtable}

\FloatBarrier
\section{Subscribing}
Tasks can wait for or poll (with possible timeout) these events, this is known as Subscribing. Events can be subscribed to either manually using \textbf{TaskWait} and \hyperref[func:TaskPoll]{\textbf{TaskPoll}}, or automatically by other system calls. A subscription requires the publisher's ID (also called the event source), the object specific event e.g. a Timer Overflow event for Timers, Event Flags to enable/disable subscription options these flags are shown in \cref{tbl:eventflags} and an optional timeout time to avoid blocking tasks for longer than necessary.\\


\begin{pdtable}{tbl:eventflags}{Event flags}
	\begin{pdtabular}{l|c|p{9.5cm}}
		\pdtablehead{\textbf{Macro}  & \textbf{Event flag} & \textbf{Description}}
		\pdtablerow{\textit{EVENT\_FLAG\_NONE} & None & Passed when no event flags should be set.}
		\pdtablerow{\textit{EVENT\_FLAG\_ADDRESSED} & Addressed & The event will be addressed to the specified
			task. This means that the task does not need a subscription to these events in order to receive them. However, addressed events cannot be handled using TASK\_EVENT\_HANDLER sections.}
		\pdtablerow{\textit{EVENT\_FLAG\_NO\_HANDLER} & No handler & Set when the subscribing task does not implement an explicit \textbf{TASK\_EVENT\_HANDLER} section for this event.}
		\pdtablerow{\textit{EVENT\_FLAG\_NO\_TRIGGER} & No trigger & Set to keep the subscribing task from being scheduled when the event was received.}
		\pdtablerow{\textit{EVENT\_FLAG\_PERMANENT}  & Permanent & Indicates that the subscription to this event
			is permanent i.e. the event subscription will not be removed after receiving it.}	
	\end{pdtabular}
\end{pdtable}

\FloatBarrier
\section{Handling}
When an event occurs and is published the subscribed tasks will receive this event and will typically, depending on the subscription flags, be scheduled to execute. Once a triggered task is executing, the task handles occurred events in so called Task Event Handler sections. These sections are defined by a \textbf{TASK\_EVENT\_HANDLER\_BEGIN} and \textbf{TASK\_EVENT\_HANDLER\_END} pair. Event timeouts can be handled in \textbf{TASK\_EVENT\_HANDLE\_TIMEOUT} sections.

\section{Characteristics}
Events offer a generic and very flexible way of communicating throughout the system. However, there are a few things to be kept in mind when designing an application:
\begin{pditemize}
	\item When a event is published, it only exists for a defined amount of OS ticks\\ (\textit{PRTOS\_CONFIG\_EVENT\_LIFE\_TIME\_TICKS}). After that the event will be destroyed, unless it is reset by another occurrence. 
	\item Any event, published or subscribed, cannot be added to the same list twice. Instead the event's lifetime is reset. 
	\item It IS possible to subscribe to new events within \textit{EVENT\_HANDLER} sections.
	\item It IS NOT possible to subscribe to the same event BEFORE its handler section. This would clear both the occurred event and the newly subscribed one.
	\item Events CANNOT be received if a task has been transitioned to the Idle state or Disabled state manually, even if it still has valid subscriptions.
	\item Events CAN still be received when a task is already scheduled and waiting to be executed.
\end{pditemize}

\TODO{Exampe of event handling}


\FloatBarrier
