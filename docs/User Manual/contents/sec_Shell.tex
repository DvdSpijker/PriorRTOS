\section{\textit{Shell} API}

\subsection{ShellCommandRegister}
\label{func:ShellCommandRegister}
\begin{pdfunction}
{ShellCommandRegister(struct ShellCommand command) }
{ 
Registers a Shell command making available for calling using 
CLI. }
\pdfunctionargin{(struct ShellCommand *) command}{ Initialized ShellCommand structure 
                                   that defines the command, callbacks and
                                   token counts.}
\pdfunctionreturn{(OsResult_t) }{sys call result }{OS_OK}{if the command was registered. }
\pdfunctionreturn{(OsResult_t) }{sys call result }{OS_FAIL}{if the maximum amount of registered commands has beenreached. }
\end{pdfunction}

\subsection{ShellReplyInvalidArgs}
\label{func:ShellReplyInvalidArgs}
\begin{pdfunction}
{ShellReplyInvalidArgs(char command) }
{ 
Should be called by the command callbacks when the contents of 
arguments are invalid. The user will be informed with this message: 
'<command>' has invalid arguments". }
\pdfunctionargin{(char *command) command}{ Command in string form. }
\end{pdfunction}

\subsection{ShellPut}
\label{func:ShellPut}
\begin{pdfunction}
{ShellPut(char message, ...); }
{ 
Prints a message starting on a new line prefixed with 'psh>'. }
\pdfunctionargin{(char *) message}{ Message in string form. }
\pdfunctionargin{(...) variable arguments}{ - }
\pdfunctionreturn{(U16_t) }{Number of characters }{0}{if no characters were printed because the buffer could not process the requested amount. }
\pdfunctionreturn{(U16_t) }{Number of characters }{Other}{valid number of characters. }
\end{pdfunction}

\subsection{ShellPutRaw}
\label{func:ShellPutRaw}
\begin{pdfunction}
{ShellPutRaw(char message, ...); }
{ 
Prints the exact message. }
\pdfunctionargin{(char *) message}{ Message in string form. }
\pdfunctionargin{(...) variable arguments}{ - }
\pdfunctionreturn{(U16_t) }{Number of characters }{0}{if no characters were printed because the buffer could not process the requested amount. }
\pdfunctionreturn{(U16_t) }{Number of characters }{Other}{valid number of characters. }
\end{pdfunction}

\subsection{ShellPutRawNewline}
\label{func:ShellPutRawNewline}
\begin{pdfunction}
{ShellPutRawNewline(char message, ...); }
{ 
Prints the exact message on a new line. }
\pdfunctionargin{(char *) message}{ Message in string form. }
\pdfunctionargin{(...) variable arguments}{ - }
\pdfunctionreturn{(U16_t) }{Number of characters }{0}{if no characters were printed because the buffer could not process the requested amount. }
\pdfunctionreturn{(U16_t) }{Number of characters }{Other}{valid number of characters. }
\end{pdfunction}
