\section{\textit{Task} API}

\subsection{TaskCreate}
\label{func:TaskCreate}
\begin{pdfunction}
{TaskCreate(Task\_t handler, TaskCat\_t category,  Prio\_t priority, void args) }
{ 
Creates a Task Control Block for the given handler. The 
is now able to be scheduled based on its category and priority. 
arguments are passed to task upon execution. }
\pdfunctionargin{(Task_t) handler}{ Task handler function. }
\pdfunctionargin{(TaskCat_t) category}{ Task category: TASK\_CAT\_LOW, TASK\_CAT\_MEDIUM, 
                                 TASK\_CAT\_HIGH, TASK\_CAT\_REALTIME.}
\pdfunctionargin{(Prio_t) priority}{ Task priority: 1-5 (5 is highest). }
\pdfunctionargin{(void*) args}{ Task arguments. Passed to the task when executing. }
\pdfunctionreturn{(Id_t) }{Task ID }{INVALID_ID}{if an error occurred during task creation. }
\pdfunctionreturn{(Id_t) }{Task ID }{Other}{if successful. }
\end{pdfunction}

\subsection{TaskDelete}
\label{func:TaskDelete}
\begin{pdfunction}
{TaskDelete(Id\_t task\_id) }
{ 
Deletes an existing Task Control Block for the given handler. 
task cannot be scheduled anymore. }
\pdfunctionargin{(Id_t *) task_id}{ Task ID. INVALID\_ID = Running task ID. }
\pdfunctionreturn{(OsResult_t) }{sys call result }{OS_OK}{if operation was successful. }
\pdfunctionreturn{(OsResult_t) }{sys call result }{OS_ERROR}{if the task handler was not found in any of the lists. }
\end{pdfunction}

\subsection{TaskDeadlineSet}
\label{func:TaskDeadlineSet}
\begin{pdfunction}
{TaskDeadlineSet(Id\_t task\_id, U32\_t t\_ms) }
{ 
Sets the scheduling deadline of the given Real-Time task. 
overrides the default deadline defined by 
CONFIG\_REAL\_TIME\_TASK\_DEADLINE\_MS\_DEFAULT}
\pdfunctionargin{(Id_t) task_id}{ Real-Time Task ID. }
\pdfunctionargin{(U32_t) t_ms}{ Deadline in ms. }
\pdfunctionreturn{(OsResult_t) }{sys call result }{OS_OK}{if operation was successful. }
\pdfunctionreturn{(OsResult_t) }{sys call result }{OS_INVALID_ID}{if the given task was not a Real-Time task or if the task ID was an invalid ID (INVALID\_ID). }
\pdfunctionreturn{(OsResult_t) }{sys call result }{OS_ERROR}{if the task handler was not found in any of the lists. }
\end{pdfunction}

\subsection{TaskPrioritySet}
\label{func:TaskPrioritySet}
\begin{pdfunction}
{TaskPrioritySet(Id\_t task\_id, Prio\_t new\_priority) }
{ 
Assigns a new priority level to the given task. }
\pdfunctionargin{(Id_t) task_id}{ Task ID. INVALID\_ID = Running task ID. }
\pdfunctionargin{(Prio_t) priority}{ New task priority: 1-5 (5 is highest). }
\pdfunctionreturn{(OsResult_t) }{sys call result }{OS_OK}{if operation was successful. }
\pdfunctionreturn{(OsResult_t) }{sys call result }{OS_OUT_OF_BOUNDS}{if the new priority was not within bounds (1-5). }
\pdfunctionreturn{(OsResult_t) }{sys call result }{OS_ERROR}{if the task handler was not found in any of the lists. }
\end{pdfunction}

\subsection{TaskPriorityGet}
\label{func:TaskPriorityGet}
\begin{pdfunction}
{TaskPriorityGet(Id\_t task\_id) }
{ 
Returns the priority level of the given task. }
\pdfunctionargin{(Id_t) task_id}{ Task ID. INVALID\_ID = Running task ID. }
\pdfunctionreturn{(Prio_t) }{task priority }{0}{task could not be found. }
\pdfunctionreturn{(Prio_t) }{task priority }{15}{valid task priority. }
\end{pdfunction}

\subsection{TaskCategorySet}
\label{func:TaskCategorySet}
\begin{pdfunction}
{TaskCategorySet(Id\_t task\_id, TaskCat\_t new\_cat) }
{ 
Assigns a new category to the given task. }
\pdfunctionargin{(Id_t) task_id}{ Task ID. INVALID\_ID = Running task ID. }
\pdfunctionargin{(TaskCat_t) new_cat}{ New Task category: TASK\_CAT\_LOW, TASK\_CAT\_MEDIUM, 
                                TASK\_CAT\_HIGH, TASK\_CAT\_REALTIME.}
\pdfunctionreturn{(OsResult_t) }{sys call result }{OS_OK}{if operation was successful. }
\pdfunctionreturn{(OsResult_t) }{sys call result }{OS_OUT_OF_BOUNDS}{if new\_cat is not a member of TaskCat\_t. }
\pdfunctionreturn{(OsResult_t) }{sys call result }{OS_ERROR}{if the task was not found in any of the lists. }
\pdfunctionreturn{(OsResult_t) }{sys call result }{OS_RESTRICTED}{if the OS category was assigned in user-mode. }
\end{pdfunction}

\subsection{TaskCategoryGet}
\label{func:TaskCategoryGet}
\begin{pdfunction}
{TaskCategoryGet(Id\_t task\_id) }
{ 
Returns the given task's category. }
\pdfunctionargin{(Id_t) task_id}{ Task ID. INVALID\_ID = Running task ID. }
\pdfunctionreturn{(TskCat_t) }{task category }{TASK_CAT_OS}{(reserved by OS) }
\pdfunctionreturn{(TskCat_t) }{task category }{TASK_CAT_REALTIME}{}
\pdfunctionreturn{(TskCat_t) }{task category }{TASK_CAT_HIGH}{}
\pdfunctionreturn{(TskCat_t) }{task category }{TASK_CAT_MEDIUM}{}
\pdfunctionreturn{(TskCat_t) }{task category }{TASK_CAT_LOW}{}
\end{pdfunction}

\subsection{TaskStateGet}
\label{func:TaskStateGet}
\begin{pdfunction}
{TaskStateGet(Id\_t task\_id) }
{ 
Returns the given task's current state. }
\pdfunctionargin{(Id_t) task_id}{ Task ID. INVALID\_ID = Running task ID. }
\pdfunctionreturn{(TaskState_t) }{task state }{}{}
\end{pdfunction}

\subsection{TaskRuntimeGet}
\label{func:TaskRuntimeGet}
\begin{pdfunction}
{TaskRuntimeGet(Id\_t task\_id) }
{ 
Returns the given task's average runtime in microseconds. }
\pdfunctionargin{(Id_t) task_id}{ Task ID. INVALID\_ID = Running task ID. }
\pdfunctionreturn{(U32_t) }{task runtime in us. }{0}{if the task could not be found. }
\pdfunctionreturn{(U32_t) }{task runtime in us. }{Other}{if valid. }
\end{pdfunction}

\subsection{TaskIdGet}
\label{func:TaskIdGet}
\begin{pdfunction}
{TaskIdGet(void) }
{ 
Returns the ID of the current task. }
\pdfunctionreturn{(Id_t) }{Task ID }{INVALID_ID}{error occurred. }
\pdfunctionreturn{(Id_t) }{Task ID }{Other}{valid Task ID. }
\end{pdfunction}

\subsection{TaskNotify}
\label{func:TaskNotify}
\begin{pdfunction}
{TaskNotify(Id\_t task\_id, void args) }
{ 
Notifies the task to be come active, 
it available for scheduling. 
specified arguments will be passed to the task upon execution. }
\pdfunctionargin{(Id_t) task_id}{ Task ID. }
\pdfunctionargin{(void *) args}{ Task arguments. }
\pdfunctionreturn{(OsResult_t) }{sys call result }{OS_OK}{if the task was activated. }
\pdfunctionreturn{(OsResult_t) }{sys call result }{OS_ERROR}{if the task could not be found. }
\end{pdfunction}

\subsection{TaskSleep}
\label{func:TaskSleep}
\begin{pdfunction}
{TaskSleep(U32\_t t\_ms) }
{ 
Calling task will sleep for the specified amount of 
after exiting. After the sleep-timer expires, the task is 
woken and executed. }
\pdfunctionargin{(U32_t) t_ms}{ Sleep time in milliseconds. }
\pdfunctionreturn{(OsResult_t) }{sys call result }{OS_OK}{if the sleep timer was created. }
\pdfunctionreturn{(OsResult_t) }{sys call result }{OS_ERROR}{if the task could not be found. }
\end{pdfunction}


\subsection{TaskPoll}
\label{func:TaskPoll}
\begin{pdfunction}
{TaskPoll(Id\_t object\_id, U32\_t event, U8\_t flags,  U32\_t timeout\_ms, Id\_t out\_event\_id) }
{ 
Subscribes the task to the specified event published by the
specified object in a NON-BLOCKING fashion.\\
When this event occurs the task will receive it and, depending on the
event flags passed, be activated.\\
If the event does not occur within the specified time, the event subscription
times out and the task will be activated to handle the timeout. }
\pdfunctionargin{(Id_t) object_id}{ ID of the event generating object. If INVALID\_ID 
                             the task will be subscribed to all.}
\pdfunctionargin{(U32_t) event}{ Event to subscribe to. }
\pdfunctionargin{(U8_t) flags}{ Flags to be set, use EVENT\_FLAG\_ macros. }
\pdfunctionargin{(U32_t) timeout_ms}{ Event timeout in milliseconds. If 0 is passed, 
                               the task will wait indefinitely.}
\pdfunctionreturn{(OsResult_t) }{sys call result }{OS_OK}{if the subscription was successful. }
\pdfunctionreturn{(OsResult_t) }{sys call result }{OS_ERROR}{if an error occurred. }
\end{pdfunction}

